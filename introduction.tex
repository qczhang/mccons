\documentclass[a4paper, 12pt] {article}
\usepackage[utf8]{inputenc}
\usepackage[pdftex]{graphicx}

\newcommand{\HRule}{\rule{\linewidth}{0.5mm}}

\begin {document}


%title page

\begin{titlepage}
\begin{center}
\textsc{\LARGE Université de Montréal}\\[1.5cm]

\HRule \\[0.4cm]

{\huge Multiobjective consensus sequence search
\HRule \\[0.4cm]}
\vfill
\emph{Authors:}\\
Gabriel \textsc{Parent}\\
Stefanie \textsc {Schirmer}
\vfill
\emph{Supervisor:} \\
Dr.~François \textsc{Major}
\vfill
{\large \today}
\end{center}
\end{titlepage}



% first section, the introduction
\section*{INTRODUCTION}

\noindent
\section*{USE CASE}
\subsection*{Consensus sequence search (mccons)}
\begin{enumerate}
  \item There is a family of multiple RNA sequences who share a common function.
  \item There is a reason to suspect that the fucntion is related to structure. 
  \item There is a tool which can give you different possibilities of structure (suboptimals) for each sequence.
\end{enumerate}
A simple example of use would be to try to find common structures for tRNAs, IRE or families within Rfam.

\subsection*{Distance measures comparison}
\begin{enumerate}
  \item There are multiple distance functions which are of interest.
  \item These distance functions respect at least these criterias
    \begin{enumerate}
      \item d(x, y) $\geq$ 0 (non-negativity)
      \item d(x, y) = 0 iff x = y 
      \item d(x, y) = d(y, x) (symmetry)
    \end{enumerate}
  \item There is interest in observing what tradeoffs are involved between the distance functions.
\end{enumerate}
A simple case of use could be to analyze how the base pair set distance, mountain distance and hausdorff distance
agree between themselves and what kind of structure score high in certain and low in others. It gives a sense
of what is really compared and which distance function should be used for certain tasks.\\

\noindent
\section*{Workflow (mccons)}
\subsection*{Inputs}
\noindent
The current mccons version takes two inputs:

\begin{enumerate}
  \item $n$ possible 2D structures (Vienna dot bracket) of $m$ different sequences of RNA.
  \item $j$ distance functions between theses structures
\end{enumerate}
\noindent

\subsection*{Representation}
A solution is created by randomly choosing a structure for each RNA family. In the language of
genetic algorithms, a solution is an "individual".
In $Julia$, a solution is represented by an immutable type with two fields.
\HRule \\[0.4cm]
\begin{math}
\noindent
immutable \; solution\\
\indent units::Vector\\
\indent fitness::Vector\\
end\\
\noindent
\end{math}
\HRule \\[0.4cm]
Units represents the vector of chosen structures from each family.
e.g. units[i] is a structure that belongs to family i.

Fitness, is a vector of objective values calculated for each distance, when the solution
is initialized. If there are n measures, we sum the value of the $n(n-1)/2$ comparisons. If the distance
function is not symmetric, we must symmetrize it by summing instead $n^2$ comparisons (all against all).


The vector or all solutions is called a population. In $Julia$, the population is simply a vector of
solutions.\\
\HRule \\[0.4cm]
\begin{math}
type \; population\\
\indent  individuals::Vector\{solution\}\\
end
\end{math}\\
\HRule \\[0.4cm]



\subsection*{Internal working}
\noindent
The NSGA-II algorithm is used to minimize the sum of pairwise distances between chosen RNA structures.


\subsubsection*{Initialization}
\noindent
At first, two popuations are generated. There are options as to how to choose, for example it could
be a good choice to bias the choice based on free energy in our case. The current implementation simply
chooses randomly.

\subsubsection*{Iterations}
\begin{enumerate}
  \item The two populations (old and new) are merged into a new population of size $2N$.
  \item The $2N$ population is sorted in non dominated fronts (exclusive sets). These fronts respect two properties.	
  \begin{itemize}
    \item within each front, each solution is not dominated by any other from the same front
    \item every solution of a "better" front is nondominated by every solutions that belong to "inferior" fronts
  \end{itemize}
  \item Initialize S = {}. We add the fronts to it in descending order of domination until an addition of a front would make $\left\vert{S}\right\vert \geq$ N. 
  \item We take $k = N - \left\vert{S}\right\vert $ from the last front, based on crowding distance withing the last front not added.
  \item We proceed by tournament selection to choose the individuals that will reproduce.
  \item We generate offsprings from the selected individuals and iterate again.
\end{enumerate}

\subsubsection*{Output}
The output of this search is a set of solutions which are nondominated between themselves. We achieve this by 
sorting into non dominated fronts the last population and taking the dominating front.

% Reproduce the pseudocode with the 
% https://en.wikibooks.org/wiki/LaTeX/Algorithms_and_Pseudocode
% procedure NSGA-IIR
%   Initialize P1 and Q1
%   for t = 1..NGEN
%     Rt = Pt U Qt
%     F = NONDOMINATEDSORT(Rt)
%     Pt+1 = {}
%     j = 1
%     while |Pt+1| + |Fj| <= N
%       F ={f(i) | i E Fj}
%       CROWDINGDISTANCE(F)
%       Pt+1 += Fj
%       j+=1
%     end
%     
%     k = N - |Pt+1|
%     Pt+1 = Pt+1 U LASTFRONTSELECTION(Fj, k)
%     P't+1 = UFTOURNSELECTION(Pt+1, N)
%     Qt+1 = GENERATEOFFSPRING(P't+1)
%   end
% end
 




\end{document}



 








































